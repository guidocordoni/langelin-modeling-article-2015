\documentclass{article}

\usepackage[a4paper,margin=1in]{geometry}

\usepackage{fontspec}
\setmainfont{Linux Libertine}

\usepackage{graphicx}
\graphicspath{{../figures/}{./}}

\begin{document}

\begin{center}
    \textbf{Handout for poster “Algorithmic generation of random languages argues for syntax as a source of phylogenetic information”}
\end{center}

Erratum: the bar chart should be replaced with the following one:

\includegraphics[width=4in]{close-pair-bars-poster}

\noindent{}References:

\begin{itemize}
  \item Biberauer, T., Roberts, I. 2012. “The significance of what hasn’t
    happened”, paper presented at DiGS 14 (Lisbon).
  \item Bortolussi, L., Longobardi, G., Guardiano, C., Sgarro,
    A. 2011. How many possible languages are there? In Bel-Enguix, G.,
    Dahl, V., Jiménez-López, M. D. (eds). \emph{Biology, Computation and
    Liguistics}, 168–179. Amsterdam, IOS Press.
  \item Bouckaert, R. et al. 2012. Mapping the origins and expansion of
    the Indo-European language family. \emph{Science} 337, 957–960.
  \item Clark, R., Roberts, I. 1993. A computational model of language
    learnability and language change. \emph{Linguistic Inquiry} 24, 299–345
  \item Dunn, M., Greenhill, S. J., Levinson, S. C., Gray, R. D.,
    2011. Evolved structure of language shows lineage-specific trends in
    word-order universals.  \emph{Nature} 473, 79–82.
  \item Dyen, I., Kruskal, J., Black, P. 1992. An Indo-European
    classification: a lexicostatistical experiment. \emph{Transactions
        of the American Philosophical Society}, 82(5).
  \item Gray, R., Atkinson, Q. 2003. Language tree divergences support
    the Anatolian theory of Indo-European origin. \emph{Nature}, 426:
    435–439.
  \item Greenberg, J. 1963. \emph{Universals of Language}. London: MIT Press.
  \item Guardiano, C., Longobardi, G. 2005. Parametric Comparison and
    Language Taxonomy. In Batllori, M., Hernanz, M. L., Picallo C., Roca
    F. (eds). \emph{Grammaticalization and Parametric
    Variation}. 149-174. Oxford University Press.
  \item Lightfoot, D. 2006. \emph{How new languages emerge}. Cambridge
    University Press.
  \item Ringe, D., Warnow, T., Taylor, A. 2002. Indo-European and computational
    cladistics. \emph{Transactions of the Philological Society}, 100 (1):
    59–129.
  \item Wexler, K. 2011. “Grammatical computation in the Optional
    Infinitive Stage”. In \emph{Handbook of Generative Approaches to
        Language Acquisition}, ed. de Villiers, J.~ and Roeper,
    T. 53–118. New York, NY: Springer.
\end{itemize}


\end{document}
